%%%%%%%%%%%%%%%%%%%%%%%%%%%%%%%%%%%%%%%%%
% Cheatsheet
% LaTeX Template
% Version 1.0 (12/12/15)
%
% This template has been downloaded from:
% http://www.LaTeXTemplates.com
%
% Original author:
% Michael Müller (https://github.com/cmichi/latex-template-collection) with
% extensive modifications by Vel (vel@LaTeXTemplates.com)
%
% License:
% The MIT License (see included LICENSE file)
%
%%%%%%%%%%%%%%%%%%%%%%%%%%%%%%%%%%%%%%%%%

%----------------------------------------------------------------------------------------
%	PACKAGES AND OTHER DOCUMENT CONFIGURATIONS
%----------------------------------------------------------------------------------------

\documentclass[11pt]{scrartcl} % 11pt font size

\usepackage[utf8]{inputenc} % Required for inputting international characters
\usepackage[T1]{fontenc} % Output font encoding for international characters

\usepackage[margin=0pt, landscape]{geometry} % Page margins and orientation

\usepackage{graphicx} % Required for including images

\usepackage{color} % Required for color customization
\definecolor{mygray}{gray}{.75} % Custom color

\usepackage{url} % Required for the \url command to easily display URLs

\usepackage[ % This block contains information used to annotate the PDF
colorlinks=false, 
pdftitle={Cheatsheet}, 
pdfauthor={John Smith}, 
pdfsubject={Compilation of useful shortcuts}, 
pdfkeywords={Random Software, Cheatsheet}
]{hyperref}

\setlength{\unitlength}{1mm} % Set the length that numerical units are measured in
\setlength{\parindent}{0pt} % Stop paragraph indentation

\renewcommand{\dots}{\ \dotfill{}\ } % Fills in the right amount of dots

\newcommand{\command}[2]{#1~\dotfill{}~#2\\} % Custom command for adding a shorcut

\newcommand{\sectiontitle}[1]{\paragraph{#1} \ \\} % Custom command for subsection titles

%----------------------------------------------------------------------------------------

\begin{document}

\begin{picture}(297,210) % Create a container for the page content

%----------------------------------------------------------------------------------------
%	TITLE SECTION 
%----------------------------------------------------------------------------------------

\put(10,200){ % Position on the page to put the title
\begin{minipage}[t]{210mm} % The size and alignment of the title
\section*{Calculator Manual -- Cheat Sheet} % Title
\end{minipage}
}

%----------------------------------------------------------------------------------------
%	FIRST COLUMN SPECIFICATION
%----------------------------------------------------------------------------------------

\put(10,180){ % Divide the page
\begin{minipage}[t]{85mm} % Create a box to house text

%----------------------------------------------------------------------------------------
%	HEADING ONE
%----------------------------------------------------------------------------------------
\sectiontitle{RESUME}

The \textbf{lex} commands generates a lexical analyzer program that analyzes input and breaks it into tokens, such as numbers, letters, or operators. The tokens are defined by grammar rules set uo in the \textbf{lex} specification. \\
The \textbf{yacc} generates a parser that analyzes input using the tokens identified by the lexical analyzer and performs specified actions, such as flagging improper syntax. Together these commands generate a lexical analyzer and parser program for interpreting input and output handling.\\

%----------------------------------------------------------------------------------------
%	HEADING TWO
%----------------------------------------------------------------------------------------

\sectiontitle{Basic Operations}
			
expression : expression PLUS expression\\
\phantom{x}\hspace{4.5em}| expression MINUS expression\\
\phantom{x}\hspace{4.5em}| expression TIMES expression\\
\phantom{x}\hspace{4.5em}| expression DIVIDE expression\\

\command{SUMA}{1 + 1}
\command{RESTA}{1 - 1}
\command{MULTIPLICACION}{1 * 1}
\command{DIVISION}{1 / 1}

%----------------------------------------------------------------------------------------
%	HEADING THREE
%---------------------------------------------------------------------------------------
\sectiontitle{MOD}

expression : expression MOD expression\\

\command{MOD OF}{$2\phantom{x.}\%\phantom{x}2$}

%----------------------------------------------------------------------------------------

\end{minipage} % End the first column of text
} % End the first division of the page

%----------------------------------------------------------------------------------------
%	SECOND COLUMN SPECIFICATION 
%----------------------------------------------------------------------------------------

\put(105,180){ % Divide the page
\begin{minipage}[t]{85mm} % Create a box to house text

%----------------------------------------------------------------------------------------
%	HEADING FOUR
%----------------------------------------------------------------------------------------

\sectiontitle{POWER}

expression : expression POW expression\\

\command{POWER OF}{$2\phantom{x}\land\phantom{x}2$}

\sectiontitle{FUNCTIONS}

expression : \\
\phantom{xxxx} FUNCTION LPAREN expression RPAREN\\

\command{\phantom{$\$$}SINE}{sen(1)}
\vspace{0.5em}
\command{COSINE}{cos(1)}
\vspace{0.5em}
\command{TANGENT}{tan(1)}
\vspace{0.5em}
\command{INVERSE TANGENT}{invtan(1)}
\vspace{0.5em}
\command{INVERSE SINE}{invsen(1)}
\vspace{0.5em}
\command{INVERSE COSINE}{invcos(1)}
\vspace{0.5em}
\command{HIPERBOLIC TANGENT}{tanh(1)}
\vspace{0.5em}
\command{HIPERBOLIC COSINE}{cosh(1)}
\vspace{0.5em}
\command{HIPERBOLIC SINE}{senh(1)}
\vspace{0.5em}
\command{INVERSE HIPERBOLIC SINE}{asenh(1)}
\vspace{0.5em}
\command{INVERSE HIPERBOLIC COSINE}{acosh(1)}
\vspace{0.5em}
\command{INVERSE HIPERBOLIC TANGENT}{atanh(1)}
\vspace{0.5em}
\command{LOGARITM BASE 10}{log10(1)}
\vspace{0.5em}
\command{LOGARITM BASE 2}{log2(1)}
\vspace{0.5em}
\command{SQUARE ROOT}{sqrt(1)}
\vspace{0.5em}
\command{NATURAL LOGARITM}{ln(1) | nlog(1)}

%----------------------------------------------------------------------------------------

\end{minipage} % End the second column of text
} % End the second division of the page

%----------------------------------------------------------------------------------------
%	THIRD COLUMN SPECIFICATION 
%----------------------------------------------------------------------------------------

\put(200,180){ % Divide the page
\begin{minipage}[t]{85mm} % Create a box to house tex

\sectiontitle{SET OPERATIONS} % Heading five

\command{\phantom{.}DEFINE A SET}{A=\{1,2,3\}}
\vspace{1mm}
\command{DEFINE A SET}{B=\{3,4,5\}}
\vspace{1mm}
\command{UNIVERSE \textbf{just type UNI}}{\{1,2,3,4,5\}}
\vspace{1.5mm}
\command{INTERSECTION}{$A \cap B$}
\command{\phantom{x}}{\{3\}}
\vspace{1.5mm}
\command{UNION}{$A \cup B$}
\command{\phantom{x}}{\{1,2,3,4,5\}}
\vspace{1.5mm}
\command{SYMMETRIC DIFFERENCE}{$A g B$}
\command{\phantom{x}}{\{1,2,4,5\}}
\vspace{1.5mm}
\command{DIFFERENCE}{$A \backslash B$}
\command{\phantom{x}}{\{1,2\}}
\vspace{1.5mm}
\command{COMPLEMENT}{$A'$}
\command{\phantom{x}}{\{4,5\}}
\vspace{1.5mm}
\command{EMPTY SET}{$\emptyset$}
\command{\phantom{x}}{\{$\emptyset$\}}


%----------------------------------------------------------------------------------------

%----------------------------------------------------------------------------------------
%	IMPORTANT FILES
%----------------------------------------------------------------------------------------

\sectiontitle{Important files}

\texttt{have a python3 version}

\texttt{install ply package with \$pip install ply}

\texttt{Run the calculator \$python calc.py}

\vspace{\baselineskip} % Whitespace before the next section

%----------------------------------------------------------------------------------------
%	LINKS AND INFORMATION
%----------------------------------------------------------------------------------------

\sectiontitle{Links and information}

\url{https://www.dabeaz.com/ply/}

\url{https://www.dabeaz.com/ply/ply.html}

%----------------------------------------------------------------------------------------
%	FOOTNOTE
%----------------------------------------------------------------------------------------

\vspace{\baselineskip}
\linethickness{0.5mm} % Thickness of the footer line
{\color{mygray}\line(1,0){30}} % Print the line with a custom color

\footnotesize{
Created by Luis Ballado, 2022\\ 
\url{https://luis.madlab.mx/}\\
				
Released under the MIT license.
}

\end{minipage} % End the third column of text
} % End the third division of the page
\end{picture} % End the container for the entire page

\end{document}
